\documentclass[11pt]{article}
\bibliographystyle{siam}

\title{\textbf{Team Zeta Project Proposal}\\
Visual Object Prediction by Machine Learning: \\ 
Object Category Approximation Using Non-Maximal Voxels}
\author{
  Chen, Tzu-Chieh\\
  \texttt{tcchenbtx}
  \and
  Ho, Edith\\
  \texttt{edithhcw}
  \and
  Marediya, Zubair\\
  \texttt{zubair-marediya}
  \and
  Tran, Mike\\
  \texttt{miketranx4}
  \and
  Zhang, Dongping\\
  \texttt{dpzhang}
}

\begin{document}
\maketitle

The paper, \textbf{\emph{Distributed and Overlapping Representations of Faces
and Objects in Ventral Temporal Cortex}}\cite{object_rec_main}, collected data 
from a sample of six subjects of which consists of five females and one male. 
It presented eight categories of stimuli, faces, houses, chairs, animals, 
shoes, bottles, tools, and control (phase-scrambled images), to each subject, 
and wanted to answer the question: if we present different categories of 
picture stimulus to a subject, whether each stimulus category would evoke the 
same of category-specific pattern of response in the ventral object vision 
pathway, and whether we could distinguish individual categories from the 
patterns of response evoked. Each subject were placed into the functional 
magnetic resonance imaging (fMRI) for 12 times thus generating 12 time series 
data. For each time of scanning, it began with 12 seconds of rest, followed by 
eight stimulus blocks of 24-s duration, one for each category and separated by 
12-s intervals of rest, and ended the procedure with another 12 seconds of 
rest. During each 24-s stimulus blocks, stimuli were presented for 500 ms 
followed by an interstimulus interval for 1500 ms, thus presented a total of 12 
stimuli during each stimulus block and 96 stimuli in a complete experiment run.\\

The data collected for each subject were split into two sets: odd runs and even 
runs. Correlation were used as indices of response similarity, and from 
analyzing within-category correlation and between-category correlation, the 
result suggested that category-specific patterns of response were distributed 
and overlapping. The result brought a new question of whether each stimulus 
category evoked a distinct pattern of response in cortex that responded 
maximally to other categories. To test whether the patterns of non-maximal 
responses carry category-related information, voxels that responded maximally 
to either category were excluded from calculation of correlations. It turned 
out that removal of maximally responded voxels from correlation calculation 
barely diminished the accuracy of identification. The research reached a 
conclusion that the pattern of large and small responses, not just the location 
of large responses, carry category-related information, and small responses 
are an integral part of the representation.\\

The dataset for this paper could be found on OpenfMRI database and is curated 
and available for download. This 2 GB dataset (ds105) contains files that 
describe the details of this study: general information of this study 
(README file), research articles associated with this dataset (references.txt), 
detail information and update for this released dataset 
(release\textunderscore history.txt), the MR repetition time for this study 
(scan\textunderscore key.txt), the name of this study 
(study\textunderscore key.txt), the major task for this study (object viewing) 
(task\textunderscore key.txt). Besides, the models folder contains two files 
showing the key conditions (the list of the object category) for this study 
(condition\textunderscore key.txt) and the comparison setting in this study 
(tast\textunderscore contrasts.txt). On top of these, each individual subject 
has his/her own directory to store the detail results. In each of these 
directories, there are four sub-directories: anatomy, behav, BOLD and model. 
The anatomy sub-directory contains the high resolution scanning for the head of 
the subject (highres001.nii.gz), the mask to get the ``brain only'' portion 
from the scan (highres001\textunderscore brain\textunderscore mask.n\\ii.gz) 
and the ``brain only'' anatomy result (highres001\textunderscore brain.nii.gz). 
The behav sub-directory does not contain information since subject behavior 
does not contribute to this study. The model sub-directory contains detail 
information, especially the onset time in seconds starting at zero, duration 
and weighting for each conditions (object category) for the 12 task runs in 
this study. The BOLD sub-directory contains the fMRI results for all the 12 
task runs from this specific subject. In each task run directory, we can find 
the fMRI result (bold.nii.gz) and a QA sub-directory containing the time series 
analysis report and the preprocessing of the fMRI results during each task run, 
nii files representing the visual of the brain while looking at specific 
objects, and confound files for the fMRI result in this study.\\ 

The approach we are taking to explore the data is as follows. First, we will
pre-process and clean the data in order to produce a dataset for easier
manipulation. Then, we will be conducting statistical analysis to find if there
is interesting correlations among the subjects and their scans. For example,
we can study whether the fMRI pattern for different categories are
significantly different between each other, within individual subjects, and
across different subjects. The Kruskal-Wallis test will be used for this
particular example. In addition, we will utilize machine learning methods to
build models predicting fMRI blood flow, hence determining what objects people
are seeing based on past training data. Some regression techniques that are
available to use for fMRI data include ridge or lasso regression, Principal
Component Analysis, or regression trees.\\

\bibliography{project}
\end{document}
